% Anhang A - Quellcode

\chapter[Erstellter Programmcode]{Erstellter Programmcode}
%
F�r das Einf�gen erstellten Programmcodes bietet sich das Paket \textsl{listings} an. Der Funktionsumfang des Paketes ist sehr gro�, weshalb hier nur ein Beispiel gezeigt werden soll, dass mit den folgenden Einstellungen erzeugt wurde:
{\lstset{language=[LaTeX]{TeX}, basicstyle=\small, keywordstyle=\color{DarkBlue}}
\begin{lstlisting}
\lstset{language=Matlab, basicstyle=\small, xleftmargin=15pt,
        keywordstyle=\color{DarkBlue}, numbers=left, 
        numberstyle=\tiny, commentstyle=\color{DarkGreen}}.
\end{lstlisting}}
%
F�r weitere Informationen zu M�glichkeiten des Pakets sei auf dessen Dokumentation verwiesen.
\section{Matlab-Code}
% set style for matlab listings
\lstset{language=Matlab, basicstyle=\small, xleftmargin=15pt,
				keywordstyle=\color{DarkBlue}, numbers=left, 
				numberstyle=\tiny, commentstyle=\color{DarkGreen}}
%
Der folgende Code dient zur Erzeugung des Bildes \ref{kap3:colorbar}. Die Einr�ckungen im Quelltext sollen lediglich die M�glichkeiten des listings-Paket zeigen.

\begin{lstlisting}
%----------------------------------------------
% Make an image that uses only 8 levels/colors.
%----------------------------------------------
imagesc(floor(8*rand(20)))

%-----------------------------------------------------------
% Make an 8-row color map from the original 64-row colormap.
%-----------------------------------------------------------
	m64 = colormap; % Beispiel Einr�ckung
m8=m64(1:8:end,:);
	colormap(m8);   % Beispiel Einr�ckung

%---------------------
% Label new color map.
%---------------------
h = colorbar % Make colorbar and save handle.
set(h, 'ytick', (1:2:16)*7/16) % Assign positions of ticks
labels = strvcat('zero','one','two','three','four', ...
'five','six','seven'); %
%Make  levels for ticks.
set(h, 'yticklabel', labels) % Assign tick labels.
\end{lstlisting}

