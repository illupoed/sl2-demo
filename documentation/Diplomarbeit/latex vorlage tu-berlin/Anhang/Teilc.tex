% Anhang A - Quellcode

\chapter[TeX-Editoren und Distributionen]{TeX-Editoren und Distributionen}

\section{TeXnicCenter}
Als frei verf�gbarer Editor f�r LaTeX-Projekte unter Windows ist TeXnicCenter in Verbindung mit der Distribution MiKTeX zu empfehlen. Hierbei sollte zuerst MiKTeX und anschlie�end TeXnicCenter installiert werden. Um die vorhandene Vorlage fehlerfrei kompilieren zu k�nnen, muss im TeXnicCenter ein bereits vordefiniertes Ausgabeprofil genutzt werden. Ausgabeprofile dienen dazu festzulegen, "`welches TeX"' (TeX, LaTeX, pdfTeX) benutzt wird und legen somit fest, in welchem Format verwendete Grafiken vorliegen m�ssen sowie das Format der Ausgabedatei. F�r weiterf�hrende Informationen sollten die entsprechenden Tutorials bem�ht werden.

\subsection{Ausgabeprofil f�r diese Vorlage}
F�r diese Vorlage kann das vordefinierte Ausgabeprofil LaTeX-->PS-->PDF genutzt werden. Auf diese Art und Weise ist es m�glich, eps-Grafiken einzubinden. Das Ergebnis des Kompilierens ist vorerst eine *.dvi-Datei. Diese wird danach automatisch zuerst in eine *.ps-Datei umgewandelt (mittels dvips.exe), welche anschlie�end in eine *.pdf-Datei konvertiert wird (ps2pdf.exe).

\subsection{Ausgabeprofile selbst Anlegen}
Manchmal kann es sinnvoll sein, eigene Ausgabeprofile zu definieren. Im folgenden wird ein Beispiel vorgestellt, mit dem ein Profil erzeugt wird, das im Projekt vor der Dokumentenklasse den Schalter "`$\backslash$ pdfoutput=0"' erwartet. Hierbei wird pdfTeX verwendet, aber eine *.dvi-Datei erzeugt, die anschlie�end nach pdf konvertiert werden soll.

Im TeXnicCenter ist unter dem Men�punkt \textit{Ausgabe} der Punkt \textit{Ausgabeprofile definieren} (auch ALT+F7) anzuw�hlen. Dort wird dann das vordefinierte Profil LATEX-->PDF kopiert und mit einem neuen, sinnvollen Namen versehen, z.\,B. Diplomarbeit. Die Registerkarten \textit{(La)TeX} und \textit{Viewer} k�nnen unver�ndert �bernommen werden. Unter \textit{Nachbearbeitung} werden nun zwei sogenannte Postprozessoren angelegt:

\begin{enumerate}

\item{DVIPS: Anwendung ist dvips.exe mit dem entsprechenden lokalen Pfad. Als Argument wird \textit{-R0 -P pdf ''\%Bm.dvi''} eingetragen.}

\item{PS2PDF: Anwendung ist gswin32c.exe (ghostview) ebenfalls mit dem entsprechenden lokalen Pfad. Als Argument wird \textit{-sPAPERSIZE=a4 -dSAFER -dBATCH -dNOPAUSE -sDEVICE=pdfwrite -sOutputFile=''\%bm.pdf'' -c save pop -f ''\%bm.ps''} angegeben.}

\end{enumerate}

Nun muss das soeben angelegte Profil noch aktiviert werden (Menu \textit{Auswahl/Aktives Ausgabeprofil w�hlen}). 

Der Vorlage liegt ein entsprechendes Beispielprofil (Datei TexnicCenterProfil.tco) bei, das importiert werden kann. Die lokalen Pfade zu den Anwendungen und Viewern m�ssen allerdings �berpr�ft bzw. per Hand erg�nzt werden.

\section{Weitere Editoren/Distributionen}
Es existiert eine Vielzahl weiterer Editoren zum Erstellen von TeX-Dokumenten. Beispiele hierf�r sind WinEdt, Winshell, LaTeX Editor oder auch LyX (wysiwyg). Eine �hnlich gro�e Auswahl herrscht bei TeX-Distributionen (z.\,B. teTeX, AucTeX, emTeX...). Eine gute �bersicht hierzu ist unter \url{http://www.math.vanderbilt.edu/~schectex/wincd/list_tex.htm} zu finden.