\message{ !name(Kap1.tex)}
\message{ !name(Kap1.tex) !offset(12) }
\subparagraph{Beispieltext}
Der Austausch von Information zwischen Systemen erfolgt mittels Signalen. Das Signal kann als mathematische Funktion von einer oder mehreren unabh�ngigen Variablen (z.\,B.\ Zeit, Raumkoordinaten) verstanden werden. Als Tr�ger des Signals sind verschiedene physikalische oder auch chemische Gr��en denkbar, wie z.\,B.\ Spannung, Strom, Druck, Lichtst�rke, Stoffkonzentration oder Temperatur.

Die Aufgabe der Signalverarbeitung besteht in der Manipulation, Analyse, Interpretation und Darstellung von Signalen. Mit anderen Worten geht es darum, den Verlauf der 
Signalfunktion gezielt zu ver�ndern. Dies kann beispielsweise �ber die Ausblendung unerw�nschter, oder die Verst�rkung geforderter Frequenzanteile im Signal erfolgen. H�ufig geht es auch um die Trennung verschiedener im Signal enthaltener Anteile zur separaten Weiterverarbeitung. Die praktische Realisierung lehnt sich, insbesondere im analogen Bereich der Signalverarbeitung, stark an die physikalische Natur des Signaltr�gers (z.\,B.\ elektrische Filter, UV-Filter) an. 

Die Signalverarbeitung hat sich in den letzten Jahrzehnten in vielen Bereichen durchgesetzt und ist aus der modernen Technik nicht mehr wegzudenken. Gro�e Einsatzgebiete liegen in der Medizintechnik, Nachrichtentechnik, Automatisierung, Verfahrenstechnik, Bild- und Sprachverarbeitung, der Unterhaltungselektronik, Funk und Navigation oder auch in der Autoindustrie. Im Prinzip findet Signalverarbeitung heute in nahezu allen industriellen Bereich ihre Anwendungen.

Entwicklungsgeschichtlich bedingt haben sich zwei in ihrer Bedeutung gleichberechtigte Zweige der Signalverarbeitung herausgebildet, 

Aufz�hlung mit anderen Zeichen
\begin{itemize}
        \item[-] die analoge Signalverarbeitung und 
        \item[-] die digitale Signalverarbeitung.
\end{itemize}

\message{ !name(Kap1.tex) !offset(-19) }
